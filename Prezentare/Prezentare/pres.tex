
\documentclass{beamer}


\mode<presentation>
{
  \usetheme{Rochester}
  \setbeamercovered{transparent}
}


\usepackage[english]{babel}
\usepackage[utf8]{inputenc}
\usepackage{times}
\usepackage[T1]{fontenc}

\title[Short Paper Title] % (optional, use only with long paper titles)
{Efficient and Robust Automated Machine Learning}

%\subtitle
%{Include Only If Paper Has a Subtitle}

\author[Author] % (optional, use only with lots of authors)
{Bucevschi Alexandru\inst{1} and Stoleru Ingrid\inst{1}}
% - Give the names in the same order as the appear in the paper.
% - Use the \inst{?} command only if the authors have different
%   affiliation.

\institute[Universities of Somewhere and Elsewhere] % (optional, but mostly needed)
{
  \inst{1}
  Facultatea de Informatica\\
  Universitatea "Alexandru Ioan Cuza"
}
% - Use the \inst command only if there are several affiliations.
% - Keep it simple, no one is interested in your street address.

\begin{document}

\begin{frame}
  \titlepage
\end{frame}

\begin{frame}{Outline}
  \tableofcontents
  % You might wish to add the option [pausesections]
\end{frame}

\section{Prezentarea subiectului}

\subsection{Automated Machine Learning}

\begin{frame}{AutoML}
	\begin{center}
		\textbf{Automatizarea procesului de aplicare a abordarilor de ML in problemele de zi cu zi.}
	\end{center}
\end{frame}

\begin{frame}{AutoML}
	\begin{center}
		\begin{itemize}
			\item Detectia automata a tipurilor datelor de intrare
			\item Determinarea automata a task-ului: clusterizare, clasificare binara/ multi-class
			\item Automatizarea procesului de feature engineering (selection/extraction)
			\item Selectarea automata a modelului
			\item Optimizarea automata a hiperparametrilor
		\end{itemize}
	\end{center}
\end{frame}


\begin{frame}{Subiectul paper-ului}
	\begin{center}
		Paper-ul curent propune un nou sistem de \textbf{AutoML}, intitulat AUTO-SKLEARN.\\
		\vspace{0.5cm}
		Acesta este bazat pe \textbf{scikit-learn}:
		\vspace{0.25cm}
		\begin{itemize}
			\item machine learning library (Python)
			\item 15 clasificatori
			\item 14 metode de feature preprocessing
			\item 4 metode de data preprocessing
		\end{itemize}
	\end{center}
\end{frame}

\begin{frame}{Avantajele sistemului propus}
		\begin{center}
			Ia in calcul in mod automat performantele anterioare pe dataset-uri similare si construieste ensemble-uri din modelele evaluate la faza de optimizare.
		\end{center}
\end{frame}

\end{document}


